\documentclass[a4paper]{article}
\usepackage[margin=0.8in,footskip=0.25in]{geometry}
\usepackage{tcolorbox} %\begin{tcolorbox} \end{tcolorbox}
\usepackage{float} % Use H for pics here
\usepackage[english]{babel}
\usepackage[utf8]{inputenc}
\usepackage{amsmath}
\usepackage{graphicx}
\usepackage[colorinlistoftodos]{todonotes}
%%%%%%%table of contents set up
\usepackage{color}   %May be necessary if you want to color links
\usepackage{hyperref}
\hypersetup{
    colorlinks=true, %set true if you want colored links
    linktoc=all,     %set to all if you want both sections and subsections linked
    linkcolor=blue,  %choose some color if you want links to stand out
}
%%%%%%% end of toc setup
\title{Particle in an infinite box}

\author{Lu Ming}

\date{\today}
% no indent on new paragraph
\setlength{\parindent}{0pt}

\begin{document}
\maketitle
\tableofcontents
\section{Conventional solution}
We set $\hbar=2m=L=1$, therefore the Shordinger equation writes:
\begin{equation}
-\frac{d^2}{x^2}\psi(x)=E\psi(x)
\end{equation}

The numerical solution procedure is:
\begin{enumerate}
\item
Guess a value of energy $E$
\item
set $\psi(0)=0$ and $\psi'(0)=1$. Setting $\psi'(0)$ to other nonzero value are OK, because it just change the normalization constant.
\item
advance a step using $\psi(x+\Delta x)=\psi'(x)\Delta x +\psi(x)$, and $\psi'(x+\Delta x)=\psi'(x)-E\psi(x)\Delta x$
\item
advance until $x=1$ and check if $\psi(L)=0$, if not, guess another value of $E$.
\item
if yes, normalize the wavefunction and print $E$.
\end{enumerate}
Some details:
\begin{enumerate}
\item 
store $\psi(x)$ and $\psi'(x)$ in a array, with length $N+1$, make sure $\psi[0]=0$. The i-th step will rewrite $\psi[i+1]$ and $\psi'[i+1]$
\item
note the normalization conditon is $c^2\Delta x\sum_i\psi[i]^2=1$.
\end{enumerate}

Reference:
\begin{enumerate}
\item
\href{http://www.wired.com/2016/03/can-solve-quantum-mechanics-classic-particle-box-problem-code/}{You Can Solve Quantum Mechanics’ Classic Particle in a Box Problem With Code}
\item
\href{http://physics.stackexchange.com/questions/275319/numerically-solving-a-particle-in-a-box-problem/275320#275320}{Numerically solving a particle in a box problem}
\end{enumerate}

\end{document}